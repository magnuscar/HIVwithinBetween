\documentclass[12pt,letter]{scrartcl}




\usepackage{hyperref}
\usepackage{graphicx}

\addtokomafont{caption}{\small}
\setkomafont{captionlabel}{\sffamily\bfseries}
\setcapindent{1em}
\usepackage{lineno}
\usepackage{float}
\usepackage{verbatim}
\usepackage[round,comma,authoryear]{natbib}
\usepackage{color}
\usepackage{dsfont}                     % Indikatorfunktion, schoenere Mengensymbole
\usepackage[noblocks]{authblk}
\usepackage{pifont}
\usepackage{amsmath}
\usepackage{latexsym}                   % Beweisk�stchen

\newcommand{\ind}{\mathds{1}} %Indikatorfunktion
\newcommand{\qed}{\hfill $\Box$ \par} % Beweisende: K�stchen
\newcommand{\figsvpn}{/Users/magnus/Desktop/virpopneutra/figs_paper}
\newcommand{\dd}{\mathrm{d}}% differential d



%%% Corrections
\newcommand{\fb}[2][]{\del{#1}{\textcolor{red}{\textbf{#2}}}}
\newcommand{\cm}[2][]{\del{#1}{\textcolor{blue}{\textbf{#2}}}}
%\newcommand{\ts}[2][]{\del{#1}{\textcolor{green}{\textbf{#2}}}}
\newenvironment{comm}{\bgroup\bfseries}{\egroup}
\newcommand{\comFB}[1]{\textcolor{red}{\texttt{[FRANCESCO: #1]}}}
\newcommand{\comCM}[1]{\textcolor{blue}{\texttt{[CARSTEN: #1]}}}
%\newcommand{\comTS}[1]{\textcolor{green}{\texttt{[TANJA: #1]}}}
\newcommand{\del}[1]{\sout{#1}}




\linenumbers




%%%%%%%%%% TITLE %%%%%%%%%%%%%%
\title{The influence of viral within-host evolution on transmission chain reconstruction: a simulation study.} 

\author[1,2,3]{Francesco Bosia}
%\author[1,2]{Tanja Stadler}
\author[1,2,$\ast$]{Carsten Magnus} 

\affil[1]{Computational Evolution, Department of Biosystems Science and Engineering, ETH Zurich, Mattenstrasse 26, 4058 Basel, Switzerland}
\affil[2]{Swiss Institute of Bioinformatics (SIB), Switzerland}
\affil[3]{Current Address: Physical Chemistry, Department of Chemistry and Applied Biosciences, ETH Zurich, Vladimir-Prelog-Weg1-5\/10, 8093 Zurich, Switzerland}




%%%%%%%%%%%%%%%%%%%%%%%%%%%%



\begin{document}

  \maketitle

\begin{abstract}
Two points are important in this paper:
\begin{enumerate}
	\item description of within-host simulation tool
	\item identify potential factors that could influence the accuracy of transmission tree reconstruction\\
\end{enumerate}
Structure for the abstract:
\begin{item}
	\item Introduction to topic
	\item Research Questions
	\item Methods
	\item Results
	\item Relevance
\end{item}

\end{abstract}

{\bf Keywords}: HIV-1 transmission, phylogenetics, transmission chain reconstruction, within-host evolution, \ldots


\section{Introduction}

\begin{itemize}
	\item Use of phylogenetics in order to estimate transmission history in communities of HIV infected individuals
	\item However: within-host evolution might play an important role in the accuracy of transmission chain reconstruction
	\item Thus: We designed a within-host evolution tool
	\item Research questions:
	\begin{enumerate}
		\item Can transmission chains be reconstructed when sampling times do not match the order of transmission events?
		\item Which effect do high within-host mutation rates have on the reliability of transmission chain reconstruction?
		\item Are these two effects dependent on which strains are transmitted (early - i.e. including latent compartment -, or random - i.e. every strain could be transmitted according to their frequencies?
	\end{enumerate} 	
\end{itemize}





\section{Materials and Methods}

\comCM{Here we describe the simulation tool, the sampling structure, the tree comparison}


\section{Results}

\section{Discussion}



\end{document}